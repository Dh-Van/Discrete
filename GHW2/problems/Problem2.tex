\documentclass[11pt]{article}
\usepackage{../../discrete}

\begin{document}

\section*{Problem 2}

\textbf{Question: }

Give a Jeopardy-style combinatorial proof of: $\sum_{k=1}^{n}k\binom{n}{k} = n2^{n-1}$

\textbf{Jeopardy: }

How many ways can you form a non-empty team from {n} people, where one person in the team must be a captain.

\textbf{LHS: }

$\binom{n}{k} \equiv \frac{n!}{k!(n-k)!}$ tells us the number of ways to form teams of $k$ from $n$ people. There are $n!$ ways to line up $n$ people. To count all teams of $k$, we can pick the first $k$ people from each distinct line. This leads to major overcounting, since the same group of $k$ people could be a different order, and thus we would overcount the same team. For each distinct line, there are $k!$ ways the team could be ordered, and $(n-k)!$ ways the people left out could be ordered. Since this is for each distinct line, we have to divide $n!$ by $k!(n-k)!$ to 'unorder' the line, which gives us the number of ways to form teams of $k$ from $n$ people.

Now we must pick a captain from the team of $k$. This is exactly $k$, since only one person can be captain out of the entire team.

To count all possible teams we can form from $n$ people, we can change $k$, and use Sum Rule to add up all the different possiblities. We set $k=1$, since the team MUST be non-empty:

$k = 1$ $\rightarrow$ Team of 1 person $\rightarrow 1 \cdot \binom{n}{1} +$

$k = 2$ $\rightarrow$ Team of 2 people $\rightarrow 2 \cdot \binom{n}{2} +$

$\dots$

$k = n$ $\rightarrow$ Team of {n} people $\rightarrow n \cdot \binom{n}{n}$


We can rewrite this as a sum: $\sum_{k=1}^{n}k\binom{n}{k} $

\textbf{RHS: }\\
Another way to approach this problem, is to first pick the captain first and then decide the team. We can do this because we know that the team MUST be non-empty, and there MUST be a captain. Because of this, there are always $n$ ways to pick a captain from $n$ people, no matter the team size.

Each person (aside from the captain) has 2 options: ON the team, OFF the team. Since each person has an independent choice, we can use the Product Rule to represent this: $2 \cdot 2 \cdot 2 \cdot $ $ \dots $ $ \cdot 2$ The captain only has 1 option: On the team, so the above multiplication is only repeated $n-1$ times.

Picking a captain and forming teams (excluding the captain) are also independent choices, so we can once again use the Product Rule to simplify the expression to: $n2^{n-1}$

\textbf{Proof: }
Since the LHS and RHS both count the exact same thing, they MUST be equal
\end{document}