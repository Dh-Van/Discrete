\documentclass[11pt]{article}

\usepackage{../discrete}

\begin{document}

\title{Group HW \#2}
\author{Michael Ku, Dhvan Shah, Pranav Bonthu}
\maketitle

\subfile{problems/Problem1.tex}
\begin{itemize}
    \item \textbf{Who Contributed:} Dhvan Shah was the main contributor, doing the formal write-up of the solution. Pranav Bonthu and Michael Ku compared their solutions to Dhvan's and provided feedback and edits.
    \item \textbf{Resources}: The textbook
    \item \textbf{Main points:} Based on group discussions, we felt that the main point of this problem was to prove the fact that functions that map from a bigger set to a smaller set cannot exist.
\end{itemize}
\subfile{problems/Problem2.tex}
\begin{itemize}
    \item \textbf{Who Contributed:} Dhvan Shah was the main contributor, doing the formal write-up of the solution. Pranav Bonthu and Michael Ku compared their solutions to Dhvan's and provided feedback and edits.
    \item \textbf{Resources}: No resources were used.
    \item \textbf{Main points:} Based on group discussions, we felt that the main point of this problem was to develop familiarity with Jeopardy-style proofs. We had to understand how to set up a question to prove LHS = RHS
\end{itemize}
\subfile{problems/Problem3.tex}
\begin{itemize}
    \item \textbf{Who Contributed:} Pranav Bonthu was the main contributor, doing the formal write-up of the solution. Dhvan Shah and Michael Ku compared their solutions to Pranav's and provided feedback and edits.
    \item \textbf{Resources}: No resources were used.
    \item \textbf{Main points:} Based on group discussions, we felt that the main point of this probelm was to understand set proofs, and different ways to approach solving them. The solution we ended up going with was the easiest and most concise.
\end{itemize}
\subfile{problems/Problem4.tex}
\begin{itemize}
    \item \textbf{Who Contributed:} Pranav Bonthu was the main contributor, doing the formal write-up of the solution. Dhvan Shah and Michael Ku compared their solutions to Pranav's and provided feedback and edits.
    \item \textbf{Resources}: No resources were used.
    \item \textbf{Main points:} Based on group discussions, we felt that the main point of this problem was understanding how to solve counting problems by breaking down the problem into simpler, mutually exclusive cases. Then using sum rule to add up all the possibilities.
\end{itemize}
\subfile{problems/Problem5.tex}
\subfile{problems/Problem6.tex}
\end{document}