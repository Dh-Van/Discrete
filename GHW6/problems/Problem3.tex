\documentclass{article}
\usepackage{discrete}
\begin{document}
\section*{Problem 3}
\textbf{Question:}\\
\textbf{Part (a):}\\
Recall your inclusion/exclusion knowledge to find an expression for the number of onto functions from S (s elements) to a set T (with some t elements). To get full credit, you cannot simply use the formula from the book. You must set up a big set A that contains all functions from S to T, and then carefully define subsets $A_i$ that contain "bad" (not onto) functions, and then explain how you are applying inclusion/exclusion to those subsets. Put your final answer in the form $"t^s - .....".$\\\\
\textbf{Solution:}\\\\
To solve this question, we will use the principle of inclusion-exclusion to take the set of all functions from $S$ to $T$ and then remove the ones that are not onto. We start by defining our big set $A$ as the set of all functions from $S$ to $T$. Each element in $S$ has $t$ possible choices in $T$, so there are $t^s$ total functions in $A$.
Now, we define subsets $A_i$ for each element $i$ in $T$, where each $A_i$ represents the set of “bad” functions that do not map any element of $S$ to $i$. These functions are not onto because they completely miss the element $i$ in $T$. What we want, then, is to count all the functions that are not in the union of these bad sets, or $|A| - |\bigcup_i A_i|$. This is where inclusion-exclusion comes in.
\\ We can start by considering what happens when we prevent one value of $T$ from being used. If we block out one element of $T$, then each of the $s$ elements in $S$ only has $(t-1)$ choices. That means there are $(t-1)^s$ such functions, and since there are $\binom{t}{1}$ ways to choose which element of $T$ to block, we get $\binom{t}{1}(t-1)^s$ total functions that miss exactly one element.
\\Next, we continue this process by increasing the number of elements that we prevent from being mapped to. If we block two elements of $T$, there are $(t-2)^s$ functions that map $S$ into the remaining $t-2$ elements, and there are $\binom{t}{2}$ ways to choose which two to exclude. The same logic applies for blocking three, four, or more elements.
\\Using inclusion-exclusion, we alternate the signs each time to make sure we are not double-counting or double-removing overlapping cases. Putting this all together, the number of onto functions from $S$ to $T$ is:
\[
    t^s - \binom{t}{1}(t-1)^s + \binom{t}{2}(t-2)^s - \binom{t}{3}(t-3)^s + \cdots + (-1)^t \binom{t}{t}(t-t)^s.
\]
This expression starts with all possible functions $t^s$, subtracts those that miss at least one element of $T$, adds back those that miss at least two, and continues this alternating pattern to correctly count all onto functions from $S$ to $T$. This can also be written as:
\[
    f_t = \sum_{i=0}^{t} (-1)^i \binom{t}{i} (t - i)^s.
\]
\textbf{Part (b):}\\
Find a recurrence relation to count $f_t$ = the number of onto functions from S  with s elements to a set T with t elements. Although philosophically you could recur on either or both s and t, I want you to find a recurrence relation that assumes that s is fixed and t is the variable that can vary.  This means that you can recur on t but not on s.  There are multiple correct RRs, but please keep working on options until you find a RR that starts with $"t^s - .....".$\\\\
\textbf{Solution:}\\
Based on the previous part, we can say that the number of onto functions from $S$ to $T$ is given by the inclusion–exclusion formula:
\[
    f_t = \sum_{i=0}^{t} (-1)^i \binom{t}{i} (t-i)^s.
\]
To better understand how this behaves and to look for a recurrence pattern, let’s compute the first few values of $f$ by changing the value of $t$ and holding $s$ constant.
\begin{itemize}
    \item If $t = 1$:
          \[
              f_1 = 1^s = 1.
          \]
    \item If $t = 2$:
          \[
              f_2 = 2^s - \binom{2}{1}(1^s) = 2^s - 2.
          \]
    \item If $t = 3$:
          \[
              f_3 = 3^s - \binom{3}{1}(2^s) + \binom{3}{2}(1^s) = 3^s - 3(2^s) + 3.
          \]
    \item If $t = 4$:
          \[
              f_4 = 4^s - 4(3^s) + 6(2^s) - 4.
          \]
    \item If $t = 5$:
          \[
              f_5 = 5^s - 5(4^s) + 10(3^s) - 10(2^s) + 5.
          \]
\end{itemize}
Now let’s analyze the pattern among these results. Between $t=1$ and $t=2$, we see that the expression starts with $t^s$ and then subtracts a term involving $\binom{t}{1}$ multiplied by a previous power. When we move to $t=3$, we notice that $2^s$ reappears with a coefficient based on $\binom{3}{1}$, and the constant term $3$ relates to $\binom{3}{2}$ and $f_1$. Similarly, for $t=4$ and $t=5$, each new expression brings back powers that appeared in earlier $f_i$ terms, each multiplied by the corresponding binomial coefficient. \\
This observation suggests that each $f_t$ depends on the previous values $f_{t-1}, f_{t-2}, \ldots$ through descending binomial coefficients. Conceptually, every function from $S$ to $T$ either uses all $t$ elements (and is onto) or misses at least one of them. If a function misses exactly $i$ elements, there are $\binom{t}{i}$ ways to choose which are missed, and the remaining functions correspond to $f_{t-i}$. \\
Putting this reasoning together, we arrive at the following recurrence relation that looks something like this:
\[
    f_t = t^s - \binom{t}{1}f_{t-1} - \binom{t}{2}f_{t-2} - \binom{t}{3}f_{t-3} - \cdots - \binom{t}{t-1}f_1.
\]
Turning this into a more formal summation, we get a final answer that looks like this:
\[
    f_t = t^s - \sum_{i=1}^{t-1} \binom{t}{i} f_{t-i}, \quad \text{with } f_1 = 1  \text{ , } s \ge 1 \text{ and }t \ge1
\]
This recurrence starts with $t^s$ and subtracts the overcounted cases where one or more of the $t$ elements are not used, matching the pattern we observed from the earlier examples.\\\\
\textbf{Part (c):}\\
Write a sentence or two that compares and contrasts the closed-form solution that you got via inc/exc and the RR that you got that counts the same thing.  What is the major difference in how these two solutions count onto functions? \\\\
\textbf{Solution:}\\
Both the closed-form solution from inclusion–exclusion and the recurrence relation begin by considering all possible functions from $S$ to $T$ (i.e., $t^s$ total) and then subtracting the cases that are not onto. The key difference lies in how they handle overcounting. In the closed-form expression, inclusion–exclusion explicitly corrects for overlaps by alternately adding and subtracting terms to account for functions missing one or more elements of $T$. In contrast, the recurrence relation inherently incorporates this correction through previous computed values $f_i$, each of which already accounts for overlaps in smaller cases. As a result, the recurrence builds onto functions incrementally, while the closed form resolves all cases simultaneously.


\begin{itemize}
    \item \textbf{Who Contributed:} Mikey Ku was the main contributor, doing the formal write-up of the solution. Pranav Bonthu and Dhvan Shah compared their solutions to Mikey's and provided feedback and edits.
    \item \textbf{Resources}: The textbook, class notes
    \item \textbf{Main points:} Based on group discussions, we felt that the main point of this problem is to practice using inclusion-exclusion on a closed form and RR, to undertstand their similarities and differences.
\end{itemize}


\end{document}
