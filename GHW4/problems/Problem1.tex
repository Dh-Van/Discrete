\documentclass[11pt]{article}
\usepackage{../../discrete}

\begin{document}

\section*{Problem 4}


\textbf{Part A:}
\begin{figure}[H]
    \centering
    \includegraphics[width=0.55\linewidth]{P1a.pdf}
    \caption{The 17 subgraphs of $K_3$}
    \label{fig:placeholder}
\end{figure}

\textbf{Part C:}\\
In order to determine how many subgraphs with at least 1 vertex from a $K_n$ complete graph with $n$ vertices, we can start by looking at how many possible vertices can be included in each subgraph.

Since there are $n$ vertices in the graph, and each subgraph must have at least 1 vertex, a subgraph can have $n$, $n-1$, $n-2$, \ldots, 2, or 1 vertices.

To find the total amount of subgraphs of the $K_n$ complete graph, we must find how many subgraphs can be made of each combination of vertices.

For example, the subgraphs of a $K_3$ graph may have 3, 2, or 1 vertices. To find the total subgraphs of $K_3$, we need to find how many subgraphs can be made from 3 vertices, how many from 2, and how many from 1. We would then be able to find the sum by adding up each of these totals.

We need to know how many ways there are to make a subgraph, given a certain amount of vertices. To do this, we can determine how many possible edges there are, which is represented by $\binom{i}{2}$, where $i$ is the number of vertices. This represents the number of edges since an edge is a connection between 2 vertices, so the total combinations of 2 vertices are equal to the number of edges.

We also know that in each subgraph, the edges can either be present or not present. Since there are 2 options for each edge, we can say that the total possible combinations of edges is $2^{\binom{i}{2}}$.

Next, since each vertex is named, they are distinguishable. This means we need to find the combinations possible for picking $i$ vertices. This can be represented by $\binom{n}{i}$, where $n$ is the number of total vertices in the set, and $i$ is the amount of vertices in the subgraph.

This leaves us with the equation $(2^{\binom{i}{2}}) \times \binom{n}{i}$ for the amount of subgraphs that can be made with $i$ vertices.

To find the total number of subgraphs, we need to do this with every possible value between 1 and $n$ and add up the total. This gives us the equation

\[
    \sum_{i=1}^{n} \binom{n}{i} 2^{\binom{i}{2}}.
\]

\begin{itemize}
    \item \textbf{Who Contributed:} Pranav Bonth and Dhvan Shah were the main contributor, doing the formal write-up of the solution. Michael Ku compared his solutions to theirs and provided feedback and edits.
    \item \textbf{Resources}: No resources were used.
    \item \textbf{Main points:} The main goal of this question was to relate graphs back to combinatorics problems, and help us understand how to connect the two concepts together.
\end{itemize}

\end{document}
