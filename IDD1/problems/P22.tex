\documentclass[11pt]{article}
\usepackage{../../discrete}

\begin{document}
\section*{Problem 2.2}
\textbf{Question:}

After winning chips, players can queue at $20$ different cashier stations to trade their chips for cash. The cashier stations are numbered $1$ to $20$, and they each have a different cashier, so these lines are different. Suppose you have $p$ people who are going to line up themselves in front of the $20$ different cashiers, but some cashiers might get no people in line. The lines that do form in front of any cashiers are definitely ordered. Imagine that $b$ of the $p$ people are identical looking Blue Men, $c$ of the $p$ people are identically-costumed and therefore indistinguishable circus artists from Cirque du Soleil, and $d$ of them are regular people who are, of course, distinguishable from each other. To avoid any doubt: All $p$ people must be in some line to cash out, but not all cashier stations must have people lined up in front of them. How many ways can these line(s) form?

\textbf{Solution Summary:}

The number of ways these lines can form is:
$\frac{(b + c + d + 19)!}{b!c!19!}$

\textbf{Justification:}

We can solve this problem by imagining that everyone is going to stand in one long line. Everyone to the right of a cashier will be in that cashiers line, effectively making each cashier a Divider. We want there to be 20 sections in this line, which means we need to use 19 Dividers (cashiers).

We can count the number of ways to order a line of $N$ items, by doing: $N!$. For now, we can assume that everyone is distinct. The total number of items in this line are the Blue Men, the Circuit Artists, the Regular People and the Dividers: $N = b + c + d + 19$. This means that there are $N! = (b + c + d + 19)!$ ways to order this line.

However, we know that the Blue Men, the Circus Artists and the Cashiers are all identical. We can have a distinct line where the $b$ Blue Men are in specific positions. The above factorial over counts this line by $b!$, where each Blue Man could be in any of the $b$ positions. This applies to every unique pattern in the line, which means we need to divide $N!$ by $b!$: $\frac{N!}{b!}$.

We can use the above reasoning with the Circus Artists and the Cashiers as well: $\frac{N!}{b!c!19!}$

Therefore, the number of ways these lines can form is:
$\frac{(b + c + d + 19)!}{b!c!19!}$
\end{document}