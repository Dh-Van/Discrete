\documentclass[11pt]{article}

\usepackage{discrete}

\begin{document}

\textbf{Find the error and redo:}\\
How many subsets of a 100-element set have more than one element?

\textbf{Solution:} There are $\binom{100}{2}$ subsets of a 100-element set with two elements, since this counts unordered pairs of distinct elements. Similarly, there are $\binom{100}{3}$ subsets with three elements. Continuing in this way and using the sum rule (these cases are mutually exclusive), there are
\[
    \sum_{k=2}^{99}\binom{100}{k}
\]
subsets of a 100-element set that have more than one element.

\textbf{Fixing the error:}\\
A subset $B$ of a set $A$ is valid only if all elements of $B$ are also elements of $A$. Using this definition, note that a subset of a 100-element set can be the set itself. The above solution omitted the 100-element subset. To fix this, change the summation to
\[
    \sum_{k=2}^{100}\binom{100}{k},
\]
which now includes the full 100-element subset.
\textbf{Totally different approach:}\\
Think of the 100-element set $S$ as a bit string of length 100. Any subset of $S$ corresponds to a bit string where a $1$ means the element is included and a $0$ means it is not. Each position has two possibilities, so by the product rule there are $2^{100}$ subsets in total.

Now remove the subsets with exactly one element (there are 100 of them) and the empty subset (one of them). This leaves
\[
    2^{100}-100-1=2^{100}-101
\]
subsets of a 100-element set that have more than one element.

\begin{itemize}
    \item \textbf{Who Contributed:} Dhvan Shah was the main contributor, doing the formal write-up of the solution. Pranav Bonthu and Michael Ku compared their solutions to Dhvan’s and provided feedback and edits.
    \item \textbf{Resources}: No resources were used.
    \item \textbf{Main points:} Based on group discussions, we felt that this problem
\end{itemize}

\end{document}