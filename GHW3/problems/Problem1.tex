\documentclass[11pt]{article}
\usepackage{../../discrete}

\begin{document}

\section*{Problem 1}

\textbf{Question:}

How many elements are in the union of five sets if the sets contain 10,000 elements each, each pair of sets has 1,000 common elements, each triple of sets has 100 common elements, every four of the sets has 10 common elements, and there is 1 common element in all five set?

\textbf{Solution:}

Let there be sets: $S_1, S_2, S_3, S_4, S_5$, where $|S_i| = 10,000$ for $i = 1, 2, ..., 5$. We want to find $|S_1 \cup S_2 \cup S_3 \cup S_4 \cup S_5|$. To do this, we can start by defining set $A$, where $A = \{S_1, S_2, S_3, S_4, S_5\}$. We know that $|A| = 5 \cdot 10,000 = 50,000$, since each set $S_i$ containts 10,000 elements.

Now, we must account for overcounting. By combining all the sets together, we have overcounted any elements in at least 2 sets. We can define subset $A_1$, such that it counts all the common elements between set $S_1$ and set $S_2$. From the problem definition, we know that $|A_1| = 1,000$, since each pair of sets has 1,000 common elements. We must do this for all possible combinations of the five sets: $S_1, ..., S_5$. To do this, we need to figure out the number of ways to pick two sets from five, which can be expressed as $\binom{5}{2}$. For whatever pair of sets we pick, the cardinality will always be 1,000. We can write this as $1,000 \cdot \binom{5}{2}$. We can now subtract this value from the cardinality of A: $50,000 - 1,000 \cdot \binom{5}{2}$ to account for overcounting

That subtraction, oversubtracts elements that are in at least 3 sets. We can define subset $A_2$, such that it counts all the commont elements between sets $S_1, S_2 and S_3$. From the problem definition, we know that $|A_2| = 100$, since each triple of sets has 100 common elements. We must do this for all possible cominations of the five sets: $S_1, ..., S_5$. To do this, we need to figure out the number of ways to pick three sets from five, which can be expressed as $\binom{5}{3}$. For whatever triple of sets we pick, the cardinality will always be 100. We can write this as $100 \cdot \binom{5}{3}$. We now add this value to the previous expression: $50,000 - 1,000 \cdot \binom{5}{2} + 100 \cdot \binom{5}{3}$

We use a similar process for elements that are in at least 4 sets. Define subset $A_3$ for elements in sets $S_1, ..., S_4$. We know $|A_3|=10$, since every four sets have 10 common elements. We need to pick four sets out of five: $\binom{5}{4}$ and multiply by the cardinality: $10 \cdot \binom{5}{4}$. Since we overcounted elements in 4 sets, we need to subtract this value from the above expression: $50,000 - 1,000 \cdot \binom{5}{2} + 100 \cdot \binom{5}{3} - 10 \binom{5}{4}$

We again use a similar process for elements that are in at least 5 sets. Define subset $A_4$ for elements in sets $S_1, ..., S_5$. We know $|A_4|=1$, since all five sets have 1 common element. Since we oversubtracted elements in 5 sets, we need to add this value to the above expression:

$50,000 - 1,000 \cdot \binom{5}{2} + 100 \cdot \binom{5}{3} - 10 \binom{5}{4} + 1$

We can express the above expression in summation notation to be more concise:

$|S_1 \cup S_2 \cup S_3 \cup S_4 \cup S_5| = \sum_{i=1}^{5}(-1)^{i-1} \binom{5}{i} \cdot 10^{5 - i}$

\begin{itemize}
    \item \textbf{Who Contributed:} Dhvan Shah was the main contributor, doing the formal write-up of the solution. Pranav Bonthu and Michael Ku compared their solutions to Dhvan's and provided feedback and edits.
    \item \textbf{Resources}: The textbook, class notes
    \item \textbf{Main points:} Based on group discussions, we felt that the main point of this problem was to understand how to correct for overcounting and undercounting when counting the number of elements in the union of sets.
\end{itemize}



\end{document}