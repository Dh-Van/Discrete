\documentclass{article}
\usepackage{discrete}
\begin{document}
\section*{Problem 5}
\textbf{Question:}\\ 
How many ways can you distribute identical \$1 bills to bribe your four professors under the following conditions? Ben must get at least one, Alessandra must get at least two, Sam must get at least four, and Steve must get at least ten. No professor will get twenty or more, since that might seem overboard. You have sixty bills to distribute, but you don't need to distribute all of your bills. Explain all models, reasoning, and work; do not just use formulas without explaining the logic.   \\\\
\textbf{Solution:}\\\\
To solve this question, we have to look at our main conditions: 
\begin{itemize}
    \item We have 60 bills that are identical.
    \item Ben has at least 1 ($B\ge1$)
    \item Alessandra has at least 2 ($A\ge2$)
    \item Sam has at least 4 ($S\ge4$)
    \item Steve has at least 10 ($T\ge10$)
    \item No professor gets $20$ or more ($B,A,S,T\le19$)
    \item Not all bills need to be distributed
\end{itemize}
With this we can begin to figure things out. The first thing we should do is preload the minimum amounts, as that removes many impossible outcomes. We preload $1$ for Ben, $2$ for Alessandra, $4$ for Sam and $10$ for Steve. This leaves us with $60-17 = 43$ bills left to distribute.
Next we use the “sticks and stones” method to count all the ways the $43$ remaining bills can be divided between the four people if not all the bills are used. We create four buckets for the professors and one extra bucket for the bills that are not used. With these five total buckets, we have $4$ bars to be distributed between the $43$ bills, making the total number of possibilities
\[
\binom{43+4}{4}=\binom{47}{4}.
\]
However this is not the end of the question. Since teachers cannot get more than $19$ bills, we have to remove all possibilities where this occurs, while also taking overlap into account. To do this we use inclusion–exclusion. Inclusion–exclusion says we take the total possibilities and remove the cases that don’t work, while adding back overlaps. Below are the possible violations:
\begin{itemize}
    \item one professor has over 19 ($B\ge20$, $A\ge20$, $S\ge20$, or $T\ge20$)
    \item two professors each have over 19 (e.g. $B\ge20$ and $A\ge20$)
\end{itemize}
No more than two professors can receive over $19$ bills, because there are only $60$ bills in total and $17$ are already allocated to meet the minimums, leaving $43$ remaining. To give three professors at least $20$ bills each would require at least $18+16+10=44$ extra bills (for example if $T=20,\ S=20,\ A=20$), but since only $43$ remain, this situation is impossible.\\
From here we can begin to compute the counts we must subtract. For example, for $A\ge20$ we already gave Alessandra 2, so $x_A=A-2\ge18$. Preloading those extra 18 into bucket A leaves $43-18=25$ bills to distribute into the 5 buckets, so we get $\binom{25+4}{4}=\binom{29}{4}$. \\\\
We can fill this out for the other cases:
\begin{itemize}
    \item if $B\ge20$, preload $19$, leaving $43-19=24$: $\binom{28}{4}$
    \item if $A\ge20$, preload $18$, leaving $43-18=25$: $\binom{29}{4}$
    \item if $S\ge20$, preload $16$, leaving $43-16=27$: $\binom{31}{4}$
    \item if $T\ge20$, preload $10$, leaving $43-10=33$: $\binom{37}{4}$
\end{itemize}
Now we have to add back all the possibilities where two professors each get more than 19, as those were double-removed. Using the same idea:
\begin{itemize}
    \item $B\ge20$ and $A\ge20$: $43-(19+18)=6$: $\binom{10}{4}$
    \item $B\ge20$ and $S\ge20$: $43-(19+16)=8$: $\binom{12}{4}$
    \item $B\ge20$ and $T\ge20$: $43-(19+10)=14$: $\binom{18}{4}$
    \item $A\ge20$ and $S\ge20$: $43-(18+16)=9$: $\binom{13}{4}$
    \item $A\ge20$ and $T\ge20$: $43-(18+10)=15$: $\binom{19}{4}$
    \item $S\ge20$ and $T\ge20$: $43-(16+10)=17$: $\binom{21}{4}$
\end{itemize}
With this, to get our final answer, we take the total number of solutions with no max restrictions, subtract all the cases where one professor goes over, and add back the cases where two professors go over:
\[
\binom{47}{4}
-\bigl[\binom{28}{4}+\binom{29}{4}+\binom{31}{4}+\binom{37}{4}\bigr]
+\bigl[\binom{10}{4}+\binom{12}{4}+\binom{18}{4}+\binom{13}{4}+\binom{19}{4}+\binom{21}{4}\bigr].
\]
Evaluating gives $50{,}970$ ways.

\begin{itemize}
\item \textbf{Who Contributed:} Michael Ku was the main contributor, doing the formal write-up of the solution. Pranav Bonthu and Dhvan Shah compared their solutions to Michael's and provided feedback and edits.
\item \textbf{Resources}: No resources were used.
\item \textbf{Main points:} We felt that the main point of this question was applying the sticks-and-stones method while also understanding how different conditions modify the way the strategy is used, as well as how to combine it with inclusion–exclusion.
\end{itemize}
\end{document}
