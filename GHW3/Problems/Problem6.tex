\documentclass{article}
\usepackage{discrete}
\begin{document}
\section*{Problem 6}
\textbf{Question:}\\ Give a Jeopardy-style combinatorial proof of the given equation. You know several interpretations of C(n,k), which we have been writing as
(first introduced in Section 4.3), and you can revisit Section 6.6 to see the definition of derangements $D_n$ as the number of permutations of distinct objects that leave no object in its original position. Ask one question that can be answered twice, once for each side of the given equation.  Do not manipulate either side with algebra. Use well-explained sentences to justify all aspects of your answers! \\\\
\[
n! = C(n,0)D_n + C(n,1)D_{n-1} + C(n,2) D_{n-2} + \cdots + C(n,n-1)D_1 + C(n,n)D_0
\]
\textbf{Jeopardy Style Question:}\\
How many ways are there to organize n distinct books on a bookshelf?\\\\
\textbf{Solution 1:}\\
We solve this problem by classifying permutations according to how many books remain in their original positions and then summing over all such cases.
If exactly $k$ books remain fixed, we must first choose which $k$ books stay in place. This can be done in $\binom{n}{k}$ ways.\\
The remaining $n-k$ books must all move to positions different from their originals. The number of ways to do this is $D_{n-k}$, the derangement number of size $n-k$ (the number of permutations of $n-k$ objects with no fixed points).\\
Thus, the total number of permutations with exactly $k$ fixed books is
\[
\binom{n}{k}D_{n-k}.
\]
Since every permutation of $n$ books has some specific number $k$ of fixed books, summing over all possible values of $k$ from $0$ to $n$ counts every permutation exactly once:
\[
\sum_{k=0}^{n}\binom{n}{k}D_{n-k}.
\]
In particular, when $k=0$ no books remain in their original positions; the term
\[
\binom{n}{0}D_{n}
\]
counts exactly the number of complete derangements of all $n$ books (i.e.\ permutations where none of the books is in its original position).\\\\
\textbf{Solution 2:}\\
A well-known way to solve this problem is to use the factorial definition. When we place the first book out of the $n$ books on the shelf, we have $n$ options for which book goes into the first slot. After that, we are left with $n-1$ options for the second slot, then $n-2$ for the next, and so on until all the slots have been filled. Applying the product rule (multiplying the number of possibilities at each step), we obtain the pattern  
\[
n \times (n-1) \times (n-2) \times (n-3)\cdots \times (n-(n-1)).
\]  
This is exactly the definition of a factorial: we multiply by each successive integer one less than the previous until we reach $1$. Therefore, $n!$ gives the total number of ways to arrange $n$ books.  \\
\begin{itemize}
\item \textbf{Who Contributed:} MMichael Ku was the main contributor, doing the formal write-up of the solution. Pranav Bonthu and Dhvan Shah compared their solutions to Michael's and provided feedback and edits.
\item \textbf{Resources}: Textbook (section 6.6)
\item \textbf{Main points:} The main goal of this question is to examine a solution and understand how it can be applied to solving a similar problem. In addition, it encourages a deeper look into the definition of a derangement and how it can be used to approach problems in a different way.
\end{itemize}
\end{document}

